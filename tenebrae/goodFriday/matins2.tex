\buildpsalm{an--vim_faciebant--solesmes_1961}{37}{8G}

\buildpsalm{an--confundantur_et_revereantur--solesmes_1961}{39}{4-alt-A}

\buildpsalm{an--alieni_insurrexerunt--solesmes_1961}{53}{4-alt-A}

\versicle{
    In(h)sur(h)rex(h)é(h)runt(h) in(h) me(h) tés(i)tes(h) i(h)ní(g.)qui.(g.)
}{
    Et(h) men(h)tí(h)ta(h) est(h) i(h)ní(h)qui(i)tas(h) sí(g.)bi.(g.)
}

\rubric{\black{Pater noster} in silence.}


%\subsection{Lesson \theresponsory.}
\lesson

Ex tractátu sancti Augustíni Epíscopi super Psalmos

%\rubric{Lam 2:8-11}

Protexísti me, Deus, a convéntu malignántium, a multitúdine operántium
iniquitátem. Jam ipsum caput nostrum intueámur. Multi Mártyres tália passi
sunt, sed nihil sic elúcet, quómodo caput Mártyrum: ibi mélius intuémur, quod
illi expérti sunt. Protéctus est a multitúdine malignántium, protegénte se Deo,
protegénte carnem suam ipso Fílio, et hómine, quem gerébat: quia fílius hóminis
est, et Fílius Dei est. Fílius Dei, propter formam Dei: fílius hóminis, propter
formam servi, habens in potestáte pónere ánimam suam, et recípere eam. Quid ei
potuérunt fácere inimíci? Occidérunt corpus, ánimam non occidérunt. Inténdite.
Parum ergo erat, Dóminum hortári Mártyres verbo, nisi firmáret exémplo.

\responsory{re--tamquam_ad_latronem--solesmes_1961.1}{8}

%\subsection{Lesson \theresponsory.}
\lesson

%\rubric{Lam 2:12-15}

Nostis qui convéntus erat malignántium Judæórum, et quæ multitúdo erat
operántium iniquitátem. Quam iniquitátem? Quia voluérunt occídere Dóminum
Jesum Christum. Tanta ópera bona, inquit, osténdi vobis: propter quod horum
me vultis occídere? Pértulit omnes infírmos eórum, curávit omnes lánguidos
eórum, prædicávit regnum cælórum, non tácuit vítia eórum, ut ipsa pótius
eis displicérent, non médicus, a quo sanabántur. His ómnibus curatiónibus
ejus ingráti, tamquam multa febre phrenétici, insaniéntes in médicum, qui
vénerat curáre eos, excogitavérunt consílium perdéndi eum: tamquam ibi
voléntes probáre, utrum vere homo sit, qui mori possit, an áliquid super
hómines sit, et mori se non permíttat. Verbum ipsórum agnóscimus in
Sapiéntia Salomónis: Morte turpíssima, ínquiunt, condemnémus eum.
Interrogémus eum: erit enim respéctus in sermónibus illíus. Si enim vere
Fílius Dei est, líberet eum.

\responsory{re--tenebrae--solesmes_1961.1}{7}

%\subsection{Lesson \theresponsory.}
\lesson

%\rubric{Lam 3:1-9}

Exacuérunt tamquam gládium linguas suas. Non dicant Judǽi: Non occídimus
Christum. Etenim proptérea eum dedérunt júdici Piláto, ut quasi ipsi a
morte ejus videréntur immúnes. Nam cum dixísset eis Pilátus: Vos eum
occídite: respondérunt, Nobis non licet occídere quemquam. Iniquitátem
facínoris sui in júdicem hóminem refúndere volébant: sed numquid Deum
júdicem fallébant? Quod fecit Pilátus, in eo ipso quod fecit, aliquántum
párticeps fuit: sed in comparatióne illórum multo ipse innocéntior.
Institit enim quantum pótuit, ut illum ex eórum mánibus liberáret: nam
proptérea flagellátum prodúxit ad eos. Non persequéndo Dóminum flagellávit,
sed eórum furóri satisfácere volens: ut vel sic jam mitéscerent, et
desínerent velle occídere, cum flagellátum vidérent. Fecit et hoc. At ubi
perseveravérunt, nostis illum lavísse manus, et dixísse, quod ipse non
fecísset, mundum se esse a morte illíus. Fecit tamen. Sed si reus, quia
fecit vel invítus: illi innocéntes, qui coëgérunt ut fáceret? Nullo modo.
Sed ille dixit in eum senténtiam, et jussit eum crucifígi, et quasi ipse
occídit: et vos, o Judǽi, occidístis. Unde occidístis? Gládio linguæ:
acuístis enim linguas vestras. Et quando percussístis, nisi quando
clamástis: Crucifíge, crucifíge?

\responsory{re--animam_meam--solesmes_1961}{8}
